\section{Introduction}
\label{sec:intro}

A new file format is needed to improve the efficiency of these bioinformatics tasks. SLOW5 is a simple file format designed to address this \cite{slow5}. It is a tab-separated values (TSV\nomenclature{TSV}{Tab-separated values}) file with a header marked by lines beginning with a `\#', followed by the metadata and time series signal data for one nanopore read per line. An example is shown in Table \ref{tab:slow5}.

In order to read a SLOW5 file using multiple threads of execution, a corresponding SLOW5 index file is used \cite{slow5}. It is also a TSV file and begins with a single header line prepended by a `\#' to define the column names and their order. Each following line represents the location of a particular read in the corresponding SLOW5 file by storing its file offset and length in bytes. See Table \ref{tab:slow5idx} for an example.

\begin{table}[h!]
    \caption{Example of a SLOW5 index file.\label{tab:slow5idx}}
    \begin{tabular}{|*{3}{l}|}
        \hline
        \#read\_id & file\_offset & length \\
        \textit{id-0} & 120 & 8073 \\
        \textit{id-1} & 8193 & 72705 \\
        \; \vdots & \; \vdots & \;\; \vdots \\
        \textit{id-$N$} & 397774 & 56481 \\
        \hline
    \end{tabular}
\end{table}

In the current readable format, a SLOW5 file containing multiple reads uses much more storage space than its corresponding FAST5 files, each containing one read. Thus, the \textit{aim} is to efficiently compress the SLOW5 file format to a size smaller than or equal to the total size of its corresponding FAST5 files, whilst still maintaining multithreaded access to the SLOW5 file format and achieving the best possible level of performance within these constraints.
